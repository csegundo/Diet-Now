\section{Vitaliy Savchenko}
\begin{itemize}
    \item \textbf{Ingeniería de requisitos:} al ser una parte esencial, para un mayor entendimiento del proyecto, participó junto a sus compañeros de manera activa en las diferentes lluvia de ideas propuestas por los diferentes integrantes del equipo. Además, Al igual que sus compañeros redactó y analizó los diferentes casos de uso que se iban a implementar en el proyecto. Diseñó y creó los \textit{mockups} iniciales de la aplicación.
    \item \textbf{Desarrollo:}
        \begin{itemize}
            \item \textbf{Modelo de datos:} participó junto a sus compañeros en la elaboración del diagrama entidad relación de la base de datos para satisfacer los requisitos de la aplicación.
            \item \textbf{Módulo de autenticación del usuario:} participó activamente en la creación de dicho modulo y ayudó en la búsqueda de información para los diferentes métodos de autenticación y creación de un usuario.
            \item \textbf{Módulo de dieta actual:} participó activamente con la creación de dicho módulo aportando diferentes estilos para la creación de las diferentes vistas asociadas al modulo.
            \item \textbf{Módulo de gestión de usuarios:} participó activamente en la creación de dicho módulo. Fue el encargado de la funcionalidad de eliminar un deportista de la plataforma así como su gestión de errores y alertas. También fue el encargado de la creación de las diferentes gráficas existentes en el perfil del usuario.
            \item \textbf{Módulo de dietas:} participó activamente en la creación de dicho módulo.
            \begin{itemize}
                \item \textbf{Módulo de todas las dietas:} participó activamente en la creación de dicho módulo. Fue el encargado de la creación de un \textit{popup} con la información nutricional detallada de un producto obtenida vía API.
                \item \textbf{Módulo de gestión de dieta:} participó activamente en la creación de dicho módulo. Fue el responsable de la inserción de los alimentos ya sea vía API o vía manual insertando los datos en el formulario correspondiente. Otro proceso que realizó fue el correcto funcionamiento de la cámara como la gestión de los permisos Android para la implementación de un escáner encargado de detectar los códigos de barras de los alimentos y hacer la consulta a la API. Por ultimo implementó la funcionalidad de borrar un documento subido de una dieta.
            \end{itemize}
            \item \textbf{Desarrollo de API REST en Node.js:} participó de forma activa en el desarrollo de esta API REST en Node.js para su uso en la aplicación.
            \item \textbf{Desarrollo de API REST en Spring:} participó de forma activa en el desarrollo de esta API REST en Spring Boot para su uso en la aplicación.
        \end{itemize}
    \item \textbf{Memoria:} participó en la elaboración de esta memoria final así como en su creación en \LaTeX \hspace{0.5mm}. Se encargó de crear los capítulos tres, cuatro y siete así como los anexos.
\end{itemize}