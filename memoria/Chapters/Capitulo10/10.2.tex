\section{Carlos Segundo Nieto}
\begin{itemize}
    \item \textbf{Ingeniería de requisitos:} al ser una parte esencial, para un mayor entendimiento del proyecto, participó junto a sus compañeros de manera activa en las diferentes lluvia de ideas propuestas por los diferentes integrantes del equipo. Además, Al igual que sus compañeros redactó y analizó los diferentes casos de uso que se iban a implementar en el proyecto.
    \item \textbf{Desarrollo:}
        \begin{itemize}
            \item \textbf{Modelo de datos:} participó junto a sus compañeros en la elaboración del diagrama entidad relación de la base de datos para satisfacer los requisitos de la aplicación.
            \item \textbf{Módulo de autenticación del usuario:} participó activamente en la creación de dicho modulo. Principalmente se encargo de la creación y la autenticación de los deportistas vía API.
            \item \textbf{Módulo de dieta actual:} participó activamente con la creación de dicho módulo. Principalmente para este módulo se encargó de gestionar los comentarios de una dieta, así como la valoración en forma de ``Me gusta`` y ``No me gustas``, al igual que las visitas. A su vez, se encargo de la gestión de las calorías consumidas durante la semana, introduciendo la lógica y el comportamiento de la interfaz
            \item \textbf{Módulo de gestión de usuarios:} participó activamente en la creación de dicho módulo así como el cambio de los credenciales de autenticación de los diferentes usuarios. También fue el encargado de maquetar el perfil del administrador proporcionándole información útil acerca de su actividad.
            \item \textbf{Módulo de dietas:} participó activamente en la creación de dicho módulo.
            \begin{itemize}
                \item \textbf{Módulo de todas las dietas:} participó activamente en la creación de dicho módulo. Su participación más característica a la hora de desarrollar el módulo fue la del cambios de estado de publicación de la dieta deseada por el usuario así como la gestión de errores de la misma. También se encargó principalmente de la vista correspondiente a visualizar los detalles de la dieta.
                \item \textbf{Módulo de gestión de dieta:} participó activamente en la creación de dicho módulo. Principalmente fue el encargado de modificar, crear y eliminar una dieta así como insertar alimentos dentro de la dieta. Otro de sus subprocesos principales para este modulo es la característica de descargar el documento insertado por el usuario que creo la dieta.
            \end{itemize}
            \item \textbf{Desarrollo de API REST en Node.js:} participó de forma activa en el desarrollo de esta API REST en Node.js para su uso en la aplicación.
            \item \textbf{Desarrollo de API REST en Spring:} participó de forma activa en el desarrollo de esta API REST en Spring Boot para su uso en la aplicación. Fue el encargado de la integridad de datos enviados a los diferentes \textit{endpoints} realizados en esta API para administradores.
        \end{itemize}
    \item \textbf{Memoria:} participó en la elaboración de esta memoria final así como en su creación en \LaTeX \hspace{0.5mm}. Se encargó de crear los capítulos tres, cuatro y nueve así como los anexos.
\end{itemize}