\section{Organización de la memoria}
A continuación se describe de manera breve la estructura de la memoria:

\begin{itemize}
    \item \textbf{Capítulo 1:} en este capítulo se describe la motivación del trabajo, los objetivos y la estructura de la memoria.
    \item \textbf{Capítulo 2:} en este capítulo se estudian herramientas similares a la que se ha realizado en el trabajo.
    \item \textbf{Capítulo 3:} en este capítulo se describe la tecnología utilizada para implementar el proyecto.
    \item \textbf{Capítulo 4:} en este capítulo se definen los actores y casos de uso que se explicarán mediante tablas junto a sus requisitos.
    \item \textbf{Capítulo 5:} en este capítulo se explica el diagrama entidad relación en el que se ha basado el proyecto y la implementación de la base de datos.
    \item \textbf{Capítulo 6:} en este capítulo se explica la arquitectura de la aplicación y los patrones utilizados.
    \item \textbf{Capítulo 7:} en este capítulo se realiza un estudio sobre el diseño e implementación de los casos de uso más relevantes.
    \item \textbf{Capítulo 8:} en este capítulo se exponen las estadísticas aportadas por usuarios que han dado su opinión sobre el producto.
    \item \textbf{Capítulo 9:} en este capítulo se enumeran los casos de uso que se harán en un futuro explicándolos brevemente.
    \item \textbf{Capítulo 10:} en este capítulo se expone el trabajo realizado por cada uno de los autores.
    \item \textbf{Anexo I:} manual de usuario.
    \item \textbf{Anexo II:} preguntas de evaluación.

\end{itemize}
