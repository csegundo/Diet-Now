\section{Motivación}
% Cuando una persona decide realizar cualquier actividad con el objetivo de mejorar su condición física, ya sea aumentando el peso o disminuyéndolo, se da cuenta que uno de los factores más determinantes para conseguir dicho objetivo, es la planificación de una dieta sana y equilibrada.

% Para ello se pueden utilizar herramientas complementarias que permiten llevar a cabo esta planificación, seguir una dieta equilibrada y planificar las diferentes comidas que se realizan durante el día.

% Con el objetivo de ayudar a todos estos usuarios, se ha decidido desarrollar una aplicación Android, en la que se le permitirá a los usuarios seguir otras dietas creadas. Utilizando esta aplicación se pueden mejorar los hábitos alimentarios del usuario, llevando un recuento de calorías consumidas, comprender el etiquetado de los alimentos, consultar la información nutricional de los mismos o incluso elaborar dietas para ayudar al resto de usuarios.

% introducir el dominio
Todo deportista necesita acompañar su actividad con una dieta adecuada ya que la alimentación es uno de los factores de los que depende el rendimiento físico. Una dieta adecuada aporta los nutrientes necesarios para mantener un estado óptimo de salud, lo que se traduce en rendimiento. Según cómo se alimente un deportista, podrá ver como afecta la alimentación a su desempeño, mejorando el rendimiento y la recuperación, limitándolos o incluso disminuyéndolos, ya que una mala alimentación puede favorecer lesiones y la fatiga.

% plantear el problema
Actualmente, no existen aplicaciones móviles que gestionen de forma flexible las dietas que un deportista necesita seguir para llevar una alimentación sana y adecuada a su perfil. Además, las dietas existentes en internet, en su mayoría carecen de fuentes o estudios que las respalden y tampoco tienen un \textit{feedback} fiable por parte de usuarios que la han probado.

% propuesta que planteamos
Para resolver estas limitaciones, se ha decidido crear una aplicación Android que cubre las necesidades del seguimiento y control de dietas. También se puede ver el \textit{feedback} de los deportistas que han seguido y valorado la dieta, haciendo el papel de red social y ayudando a otros deportistas que busquen objetivos similares. Además, los deportistas pueden adjuntar documentos a las dietas para aportar información adicional que las respalden.