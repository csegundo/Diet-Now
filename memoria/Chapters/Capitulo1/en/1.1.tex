\section{Motivation}
% When a person decides to perform any activity with the aim of improving their physical condition, either by increasing or decreasing their weight, they realize that one of the most important factors in achieving this goal is the planning of a healthy and balanced diet.

% For this purpose, complementary tools can be used to carry out this planning, to follow a balanced diet and to plan the different meals during the day.

% With the aim of helping all these users, it has been decided to develop an Android application, in which users will be allowed to follow other diets created. Using this application, the user's eating habits can be improved, keeping a count of calories consumed, understanding food labeling, consulting the nutritional information of the food or even elaborating diets to help the rest of the users.

% introducir el dominio
Every athlete needs to accompany his or her activity with an adequate diet, since nutrition is one of the factors on which physical performance depends. An adequate diet provides the necessary nutrients to maintain an optimal state of health, which translates into performance. Depending on how an athlete eats, you can see how food affects their performance, improving performance and recovery, limiting or even decreasing them, as poor nutrition can promote injuries and fatigue.

% plantear el problema
Currently, there are no mobile applications that can flexibly manage the diets that an athlete needs to follow in order to have a healthy diet that is appropriate to his or her profile. In addition, most of the existing diets on the Internet lack sources or studies that support them and do not have reliable feedback from users who have tried them.

% propuesta que planteamos
To solve these limitations, it has been decided to create an Android application that meets the needs of monitoring and control of diets. It is also possible to see the feedback of the athletes who have followed and evaluated the diet, playing the role of a social network and helping other athletes who are looking for similar goals. In addition, athletes can attach documents to the diets to provide additional information to support them.