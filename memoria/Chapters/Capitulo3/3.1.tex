\section{Herramientas utilizadas en la parte del cliente}

\subsection{Android Studio}
\textit{Android Studio} \cite{android_studio} es el entorno de desarrollo integrado oficial para la plataforma Android. Fue el entorno que reemplazó a Eclipse para el desarrollo de aplicaciones Android. Está basado en el software IntelliJ IDEA de JetBrains y ha sido publicada de forma gratuita.

Como ocurre con la mayoría de entornos de desarrollo modernos, ofrece las herramientas necesarias para la generación del código, lo que se denomina la lógica de la aplicación. Estos entornos también ofrecen los mecanismos con los que se puede diseñar la interfaz de usuario que lucirá el desarrollo final.

Este tipo de entornos ofrecen además la posibilidad de ejecutar la aplicación en un emulador Android directamente desde el IDE. Es una herramienta potente ya que se puede visualizar la aplicación en este emulador sin la necesidad de compilar la aplicación e instalarla en un dispositivo físico, además de poder elegir la marca y tamaño de la pantalla del propio dispositivo.

Toda la parte visual de la aplicación ha sido desarrollada en Android Studio, utilizando la herramienta de diseño de vistas donde se pueden arrastrar y colocar en la posición deseada los diferentes elementos que la componen.