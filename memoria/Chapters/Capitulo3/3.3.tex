\section{Otras herramientas}
\subsection{Git}
Git \cite{git} es un software de control de versiones de código abierto y gratuito. Inicialmente fue planeado para trabajar con varios desarrolladores en el núcleo de Linux. Se trata de un rastreador de contenido que se usa principalmente para almacenar código. Git posee un sistema de control de versiones para que varios desarrolladores puedan trabajar en paralelo sobre la misma aplicación permitiéndoles revertir y regresar a una versión anterior de su código

En DietNow, este servicio se ha utilizado para trabajar en un repositorio donde almacenar las distintas versiones de la aplicación, así como para poder trabajar de forma conjunta en el proyecto.

\subsection{Trello}
Trello \cite{trello} es un software de gestión en línea basado en la metodología Kanban, que permite a los usuarios trabajar de forma colaborativa, utilizando tarjetas de trabajo en un tablero, llevando de este modo una ``línea de producción`` de tareas con sus estatus correspondientes.

Este software ha sido utilizado para la gestión del proyecto DietNow, más específicamente para la organización de las tareas de desarrollo a realizar, la asignación de cada tarea a los distintos miembros del equipo de desarrollo y el conocimiento en todo momento del progreso de cada tarea. 

\subsection{Overleaf}
Overleaf \cite{overleaf} es una herramienta de publicación y redacción colaborativa en línea que hace más eficiente el proceso de redacción, edición y publicación de documentos.

Overleaf ofrece un editor \LaTeX \hspace{0.5mm} fácil de usar, con posibilidad de colaboración en tiempo real y una vista previa cargada automáticamente en segundo plano a medida que escribe.

Ha sido utilizado, junto con el libro \basix \hspace{0.5mm} \cite{basix} de \LaTeX, para la creación de la documentación relativa al proyecto DietNow.

\subsection{Google Drive}
Google Drive \cite{google_drive} es un servicio de Google de almacenamiento de archivos en la nube.

Este servicio ha sido utilizado para almacenar toda la documentación relativa al proyecto de DietNow.

\subsection{Visual Studio Code}
Visual Studio Code \cite{vscode} es un editor de código gratuito y de código abierto desarrollado por Microsoft. Posee soporte para la depuración, control integrado de Git, resaltado de sintaxis, finalización inteligente de código, fragmentos y refactorización de código. Es altamente personalizable ya que admite la instalación de distintas extensiones, cambios de temas, atajos de teclado y/o preferencias.

En DietNow este editor ha sido utilizado para el desarrollo de las APIs en Spring y Node.js necesarias para acceder a la base de datos de alimentos de Open Food Facts y para desarrollar otras funcionalidades mediante la API de Google Firebase.