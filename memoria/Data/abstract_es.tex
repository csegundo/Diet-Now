\newpage
\chapter*{Resumen}
\noindent

% En este proyecto se describe con el mayor detalle posible una aplicación destinada al ámbito de la ingeniería del software. Al hablar con muchos deportistas se puede llegar a una conclusión de que el principal problema a la hora de realizar algún deporte o destinar parte del tiempo en realizar cualquier tipo de ejercicio es la dificultad que tiene la gente en proponerse una dieta para cumplir sus metas. El objetivo de este proyecto consiste en solventar este problema mediante la realización de una red de deportistas mediante el uso de la tecnología móvil en la que poder crear, observar y publicar dietas para que todos los usuarios registrados en la plataforma puedan realizarlas y seguirlas con una mayor facilidad posible.

% Esta Herramienta se ha desarrollado mediante la tecnología Android que mediante el uso de dos roles, administrador y deportista, serán capaces de desarrollar la actividad para la que se ha llevado a cabo. El/los administrador/es serán los encargados de gestionar todo el tema de usuarios, y tendrán todos los permisos necesarios para editar, crear y borrar cualquier elemento insertado en la aplicación mientras que los deportistas serán los miembros que disfrutaran de la aplicación permitiéndoles crear, y seguir las dietas necesarias para lograr su objetivo.

En este proyecto se describe la especificación de requisitos, diseños e implementación de una aplicación Android para la gestión de dietas para deportistas.

Todo deportista necesita complementar su actividad física con una dieta saludable. En este sentido, un problema bastante habitual es ser constante en el seguimiento de la misma. Es por ello que este trabajo se ha tratado de resolver este problema proporcionado una aplicación orientada específicamente a gestionar dietas y a facilitar al usuario el seguimiento de las mismas. La aplicación ofrece al usuario la posibilidad de crear y gestionar una serie de dietas para su seguimiento. También se brinda la posibilidad de publicar las dietas creadas por un usuario, las cuales serán visibles por otros usuarios, creando así una red de personas que pueden comentar y valorar las dietas que siguen, para dejar constancia de su eficacia y ayudar con ello al resto de personas que usan la aplicación.

Durante el seguimiento de una dieta, el usuario puede introducir la cantidad consumida de cada uno de los alimentos que componen la dieta, generando un registro de la ingesta calórica por cada día de la semana. Como complemento a todo ello, también se puede introducir la cantidad de pasos caminados durante cada día, así como el peso para que el usuario pueda ver una representación gráfica del progreso.


\begin{center}
    \textbf{Palabras clave}\\
    Android, Spring, sports, API, Node.js
\end{center}
