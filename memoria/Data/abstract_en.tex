\newpage
\chapter*{Abstract}
\noindent

% This project describes in as much detail as possible an application for the field of software engineering. When talking to many sportsmen and sportswomen, it can be concluded that the main problem when it comes to doing sports or spending part of their time doing any kind of exercise is the difficulty people have in setting a diet to meet their goals. The aim of this project is to solve this problem by creating a network of athletes through the use of mobile technology in which they can create, observe and publish diets so that all registered users on the platform can perform and follow them as easily as possible.

% This tool has been developed using Android technology that through the use of two roles, administrator and athlete, will be able to develop the activity for which it has been carried out. The administrator/s will be in charge of managing all the users, and will have all the necessary permissions to edit, create and delete any element inserted in the application while the athletes will be the members who will enjoy the application allowing them to create and follow the necessary diets to achieve their goal.

This project describes the requirements specification, designs and implementation of an Android application for diet management for athletes.

Every athlete needs to complement their physical activity with a healthy diet. In this sense, a fairly common problem is to be consistent in monitoring it. That is why this work has tried to solve this problem by providing an application specifically oriented to manage diets and to make it easier for the user to follow them. The application offers the user the possibility of creating and managing a series of diets for follow-up. It also offers the possibility of publishing the diets created by a user, which will be visible to other users, thus creating a network of people who can comment on and evaluate the diets they follow, to record their effectiveness and thus help the rest of the people who use the application.

While following a diet, the user can enter the amount consumed of each of the foods that make up the diet, generating a record of caloric intake for each day of the week. As a complement to all this, the amount of steps walked during each day can also be entered, as well as the weight so that the user can see a graphical representation of the progress.

\begin{center}
    \textbf{Keywords}\\
Android, Spring, Sports, API, Node.js
\end{center}